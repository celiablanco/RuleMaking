\section*{Appendix A: Fixed-point and stability analysis}

The fixed points for the coupled system defined by Eqs.~\eqref{eq:ydot} and \eqref{eq:Edot} satisfy $\dot y=0$ and $\dot E=0$ simultaneously. From $\dot E=0$ we obtain the environmental nullcline
\begin{equation}
E=\frac{\alpha}{\beta}\,y.
\label{eq:Enull}
\end{equation}
Substituting \eqref{eq:Enull} into $\dot y=0$ yields the three equilibria
\begin{equation}
(y_0,E_0)=(0,0), \qquad (y_1,E_1)=(1,\alpha/\beta), \qquad 
(y^*,E^*)=\left(\frac{p_c}{p_{\mathrm{eff}}},\,\frac{\alpha}{\beta}\frac{p_c}{p_{\mathrm{eff}}}\right),
\label{eq:persistent_equilibria}
\end{equation}
where
\begin{equation}
p_c=c+1-s, \qquad p_{\mathrm{eff}}=\frac{p\alpha}{\beta}.
\label{eq:thresholds}
\end{equation}
The interior equilibrium exists in the admissible region only when $0<y^*<1$, or what is the same, when $p_{\mathrm{eff}}>p_c$.

Linear stability is determined from the Jacobian of the two‑dimensional flow,
\begin{equation}
J(y,E) =
\begin{pmatrix}
(1-2y)\bigl(s - c + pE - 1\bigr) & p\,y(1-y) \\
\alpha & -\beta
\end{pmatrix}.
\label{eq:persistent_J}
\end{equation}

Evaluating \eqref{eq:persistent_J} at the boundary equilibria in \eqref{eq:persistent_equilibria} and using Eq.\eqref{eq:thresholds}, gives

\begin{equation}
J(0,0)=
\begin{pmatrix}
-p_c & 0 \\
\alpha & -\beta
\end{pmatrix},
\qquad
J(1,\alpha/\beta)=
\begin{pmatrix}
p_c - p_{\mathrm{eff}} & 0 \\
\alpha & -\beta
\end{pmatrix},
\label{eq:persistent_J_boundaries_simp}
\end{equation}

Hence $(y_0,E_0)=(0,0)$ is stable whenever $p_c>0$, and $(y_1,E_1)=(1,\alpha/\beta)$ is stable when $p_{\mathrm{eff}}>p_c$.

At the interior equilibrium $(y^*,E^*)$, direct evaluation of the trace and determinant shows that
\[
\det J(y^*,E^*)=-\alpha\,p\,y^*(1-y^*)<0,
\]
whenever it exists. The determinant is therefore always negative in the admissible regime, implying one positive and one negative eigenvalue and confirming that $(y^*,E^*)$ is a saddle point.

% \vspace{8pt}
% \noindent\textbf{Proposition (Stability of the persistent‑feedback model).}
% \emph{
% For $p_{\mathrm{eff}}<p_c$, the rule‑taking equilibrium $(0,0)$ is globally attracting in the admissible region.
% For $p_{\mathrm{eff}}>p_c$, the system is bistable, with two locally stable equilibria at $(0,0)$ and $(1,\alpha/\beta)$ separated by a saddle at $(y^*,E^*)$ given in \eqref{eq:persistent_equilibria}.
% }