\begin{abstract}
A key step toward life is the shift from chemistry that merely adapts to its surroundings to chemistry that actively reshapes them. Using a coupled population-environment feedback model, we identify the conditions under which environment-modifying activity, once present, can become self-sustaining in mixed chemical populations. When environmental conditions adjust instantaneously to population composition, environmental influence leaves no lasting trace and modifiers can spread only if initially common. By contrast, when environmental modification persists over the timescale of population change, production and decay generate heritable local conditions that give rise to bistability and alternative long-term outcomes. This persistence enlarges the basin leading to sustained environmental control, suggesting that even simple feedbacks could have allowed early chemical systems to move beyond passive adaptation and begin constructing the conditions that support their own propagation.
\end{abstract}
